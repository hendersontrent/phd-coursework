\documentclass{article}
\usepackage{graphicx}
\usepackage[export]{adjustbox}
\usepackage[a4paper, left = 1cm, right = 1cm, top = 3cm, bottom = 1.25cm]{geometry}

%% Path to images
\graphicspath{ {./images/} }

\begin{document}

\title{\vspace{-4cm}Experimental Design for the Testing of a New Golf Ball}
\author{Trent Henderson}
\date{}

\maketitle

As a premier golf company, Titleist are interested in understanding the performance of their new 2022 ball design relative to their existing ball. 
As such, the following research question was developed: \textit{Does Titleist's new ball design improve spin rates around the green for players who currently play the standard Titleist ball?} 
Titleist are particularly keen to understand the impact of their ball design for players of different handicaps\footnote{The handicap system ranges from 0 to 36, with "low handicap" including 0-10, "mid" including 11-20, and "high" including 21 and above.}. 
Since low handicap players are highly skilled, we hypothesise that the new design would make a more marked improvement in spin rates for high and mid handicap players who are more reliant on technology to aid performance.

\subsection*{Participants}

A minimum of two hundred and fifty golfers who are players of the current Titleist ball will be recruited via marketing emails from the official Titleist account, meaning they will comprise a relative convenience sample. 
The demographic data collected will include participant handicap grouping (factor with three levels: Low, Mid, High), age (integer measured in years), sex (factor with three levels: Male, Female, Other/Intersex), stiffness of golf club shaft (factor with four levels: Seniors, Regular, Stiff, Extra Stiff), grip strength as measured by a dynamometer (numeric measured in kilograms), and sit-and-reach flexibility (numeric measured in centimetres).
Participants will be required to have hit no other golf balls on the testing day.

\subsection*{Materials}

\begin{itemize}
    \item 25x unmarked current golf ball (5 for warm-up) per player
    \item 25x unmarked new golf ball (5 for warm-up) per player
    \item 1x 60 degree lob wedge per player
    \item GC Quad ball tracker with TrackMan software stationed behind an indoor golf bay
\end{itemize}

\subsection*{Procedure}

Each participant will warm up by hitting ten unmarked balls that are distributed individually in a random order (which is computerised and thus the same for all participants), where five are the current Titleist ball and five are the new ball. 
Following the warm-up, the computer will then individually randomly dispense forty unmarked balls, where twenty are the current Titleist ball and twenty are the new ball. 
Again, this randomly-determined order will be the same for all participants to avoid any potential equipment biases.
All balls will be electronically marked so that the computer software will be able to detect ball type. 
The response variable \textit{spin rate} will be measured in revolutions-per-minute via a launch monitor which can accurately track spin (and other statistics) for each shot. 
Testing will occur over the course of two weeks and even numbers of players from all three handicap groupings will be recruited for each day to counterbalance any methodological effects.

\subsection*{Statistical Analysis}

It is proposed that a generalised linear model with a gamma or lognormal link function (due to spin values only being positive) will be fit to the data, with ball type, handicap grouping, a set of control variables (age, sex, shaft stiffness, grip strength, flexibility), and an interaction term between ball type and handicap grouping will be included as predictors. 
Hypothetically, if the research question and hypothesis is supported, results similar to the simulation with 95\% confidence intervals presented in Figure~\ref{fig:expectations} would be expected.
Evidently, the relative gain in spin rate for the new ball (after controlling for other variables) is much lower for high handicappers, but is quite substantial for mid and especially low handicappers. 
More simply, the new ball's technology may help high handicappers to experience performance closer to those with higher skill.

\begin{figure}[h]
    \centering
    \includegraphics[max width=\linewidth, scale=0.53]{expectations}
    \caption{\label{fig:expectations}Hypothesised relationship between ball type and handicap grouping}
\end{figure}

\end{document}